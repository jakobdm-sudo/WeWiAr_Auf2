\documentclass[a4paper,12pt]{article}

% Pakete
\usepackage[utf8]{inputenc}
\usepackage[T1]{fontenc}
\usepackage{graphicx}
\usepackage{hyperref} % Für Links
\usepackage{geometry} % Seitenränder anpassen
\geometry{margin=2.5cm}
\usepackage{enumitem} % Für Aufzählungen

% Dokumentbeginn
\begin{document}

% Titel
\title{\textbf{Werkzeuge für das wissenschaftliche Arbeiten}\\
Python for Machine Learning and Data Science}
\date{\textbf{Abgabe: 15.12.2023}}
\maketitle
\hrule\vspace{0.5cm}

% Inhaltsverzeichnis
\tableofcontents
\newpage

% Abschnitt: Projektaufgabe
\section{Projektaufgabe}

In dieser Aufgabe beschäftigen wir uns mit Objektorientierung in Python. Der Fokus liegt auf der Implementierung einer Klasse, wobei insbesondere auch Magic Methods genutzt werden.

\begin{figure}[h!]
    \centering
    \includegraphics[width=\textwidth]{../diagram/classes_files.svg}
    \caption{Darstellung der Klassenbeziehungen.}
    \label{fig:classes}
\end{figure}

\subsection{Einleitung}
Ein Datensatz besteht aus mehreren Daten, die jeweils durch ein Objekt der Klasse \texttt{DataSetItem} repräsentiert werden. Jedes Datum besitzt:
\begin{itemize}
    \item einen Namen (Zeichenkette),
    \item eine ID (Zahl) und
    \item beliebigen Inhalt.
\end{itemize}

Mehrere Daten (Objekte vom Typ \texttt{DataSetItem}) werden in einem Datensatz zusammengefasst. Die Klasse \texttt{DataSetInterface} definiert die Schnittstelle und gibt die benötigten Operationen eines Datensatzes an. Ihre Aufgabe ist es, die Klasse \texttt{DataSet} als Unterklasse von \texttt{DataSetInterface} zu implementieren.

\subsection{Aufbau}
Die folgenden Dateien sind relevant:
\begin{itemize}
    \item \texttt{dataset.py}: Beinhaltet die Klassen \texttt{DataSetInterface} und \texttt{DataSetItem}.
    \item \texttt{implementation.py}: Hier muss die Klasse \texttt{DataSet} implementiert werden.
    \item \texttt{main.py}: Testet die Schnittstelle und Operationen von \texttt{DataSetInterface}.
\end{itemize}

\subsection{Methoden}
Die Klasse \texttt{DataSet} erfordert die Implementierung folgender Methoden:
\begin{itemize}[leftmargin=1.5cm]
    \item[\textbullet] \texttt{\_\_setitem\_\_(self, name, id\_content)}: Hinzufügen eines Datums mit Name, ID und Inhalt.
    \item[\textbullet] \texttt{\_\_iadd\_\_(self, item)}: Hinzufügen eines \texttt{DataSetItem}.
    \item[\textbullet] \texttt{\_\_delitem\_\_(self, name)}: Löschen eines Datums basierend auf dem Namen.
    \item[\textbullet] \texttt{\_\_contains\_\_(self, name)}: Prüfen, ob ein Datum mit diesem Namen existiert.
    \item[\textbullet] \texttt{\_\_getitem\_\_(self, name)}: Abrufen eines Datums über seinen Namen.
    \item[\textbullet] \texttt{\_\_and\_\_(self, dataset)}: Schnittmenge zweier Datensätze.
    \item[\textbullet] \texttt{\_\_or\_\_(self, dataset)}: Vereinigungen zweier Datensätze.
    \item[\textbullet] \texttt{\_\_iter\_\_(self)}: Iteration über alle Daten im Datensatz (mit optionaler Sortierung).
    \item[\textbullet] \texttt{filtered\_iterate(self, filter)}: Gefilterte Iteration mit einer Lambda-Funktion.
    \item[\textbullet] \texttt{\_\_len\_\_(self)}: Abrufen der Anzahl der Daten im Datensatz.
\end{itemize}

% Abschnitt: Abgabe
\section{Abgabe}

Programmieren Sie die Klasse \texttt{DataSet} in der Datei \texttt{implementation.py} zur Lösung der oben beschriebenen Aufgabe. Sie können den Code entweder lokal oder im VPL bearbeiten.

Das VPL enthält zusätzliche Testfälle und Überprüfungen, um sicherzustellen, dass Ihre Implementierung korrekt ist.

\begin{flushright}
    \hrule\vspace{0.2cm}
    \footnotesize{$^*$ Dateien befinden sich im Ordner \texttt{/code/} dieses Git-Repositories.}
\end{flushright}

\end{document}
